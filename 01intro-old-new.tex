\chapter{Východzia situácia}\label{chap:intro} 
\section{Obchodovanie na menovej burze}
\subsection{Menová burza}  
Burza je osobitný druh organizovaného trhu, na ktorom predávajúci a kupujúci uskutočňujú obchody istých zastupiteľných objektov. V našom prípade mien.
\subsection{Zarábanie na menovej burze}
Na burze sa dá zarobiť spôsobom, že kúpite lacnejšie a predáte drahšie. Je to možné vďaka tomu, že na menovej burze kurz, cena za ktorú kúpite danú menu, nie je stály. A aké veľké sú výkyvy tohto kurzu, určuje volatilita burzy.
Volatilita\cite{Volatilita} je kolísanie. Miera neistoty. Spravidla platí, že čím vyššie výnosy, tým vyššia volatilita. 
\subsubsection{Vysoká volatilita}
\begin{figure}[!hbt]
\begin{center}
\includegraphics[width=1\textwidth]{obr}
\caption{Vysoká volatilita}
\label{img:vvolat}
\end{center}
\end{figure}
Na obrázku \ref{img:vvolat}  vidíme vysoko volatilnú burzu, ktorou je bitcoin a USD. Možný zisk sa pohybuje až na úrovni 37,5 percenta za mesiac, keď nerátame poplatky. V prvom obrázku je jeden dielik 10 dolárov.   
\subsubsection{Nízka volatilita}  
\begin{figure}[!hbt]
\begin{center}
\includegraphics[width=1\textwidth]{obr2}
\caption{Nízka volatilita}
\label{img:nvolat}
\end{center}
\end{figure}
Obrázok \ref{img:nvolat} ukazuje zmeny kurzu z málo volatilnej burzy, kde je  možný zisk iba 2,95 percenta za mesiac a jeden dielik je tisícina danej meny.   
\subsubsection{Broker} 
Je niečo ako investor. Jednak sa postará o všetky formality, aby sme sa mohli na takú burzu dostať, a taktiež do nás investuje. To znamená, že niekoľko krát znásobí náš vklad aby naše zisky boli niekoľko krát väčšie. Broker taktiež dáva pozor aby sme nestratili jeho peniaze, a ak by sme mali klesnúť pod hranicu jeho peňazí zakáže nám ďalej obchodovať. \cite{ZAC}  
\section{Okrajové burzy}
Keďže chceme obchodovať bez brokera, vyhovujú nám okrajové burzy.
\subsection{Výhody} 
Medzi výhody okrajových búrz patrí aj vysoká volatilita, ktorú som predstavil vyššie.
\subsubsection{Absencia \uv{Veľkých hráčov}} 
Veľký hráči\cite{ZAC} sú hlavne veľké korporácie a bohatí investori. Za to my sme na trhu malí hráči, lebo máme malý kapitál. Na okrajových burzách nie sú veľkí hráči kvôli tomu, že táto burza má ponuky s nízkym kapitálom a na ich investíciách by sa vysoko prejavila likvidita. Pojem likvidity vysvetlím v nevýhodách.
\subsubsection{Nízke poplatky} 
Výhodou okrajových búrz sú nízke transakčné a iné poplatky. Transakčné poplatky sa pohybujú od 0,1 percenta z transakcie. 
\subsection{Nevýhody} 
\subsubsection{Veľká likvidita trhu} 
\begin{figure}[!hbt]
\begin{center}
\includegraphics[width=1\textwidth]{stamp}
\caption{\uv{Order book} na Bitstamp-e}
\label{img:ob}
\end{center}
\end{figure}
Najprv vysvetlím pojem \uv{spread}\cite{ZAC}. Spread je rozdiel medzi ponukou a dopytom. Ak je ponuka 1,5 a dopyt 1,6 spread je 0,1. Likvidita trhu vyjadruje zmenu ceny pri obchodoch. Ak je likvidita nízka, tak veľké investície dočasne výrazne zmenia kurz a naopak, ak je likvidita vysoká, veľké obchody kurz zmenia minimálne. Súvisí to s tým, aké veľké množstvo peňazí je na burze. Trh na obrázku \ref{img:ob} ma vysokú likviditu. Vidíte ponuku, dopyt a spread. Ak chce niekto spraviť obchod za väčšie množstvo peňazí, len malé percento zo sumy kúpi za aktuálny kurz, lebo nie je dostatočný počet ponúk s takouto cenou. Zvyšok kupuje stále za drahšie, až príde po miesto, kt. na obrázku vyznačujem ako \uv{čiara kúpy}. Do času reakcie trhu vznikne obrovský spread a tiež to výrazne ovplyvní kurz. Preto sa veľkým hráčom sa neoplatí obchodovať na okrajových burzách. 
\subsubsection{Väčšie riziko} 
Na okrajových burzách je väčšie riziko, že svoje investície stratíte bez vášho pričinenia. Môže sa totiž stať, že burza za noc ukončí svoje pôsobenie a prestane odpovedať. V nedávnej minulosti sa stalo, že významné okrajové burzy zo dňa na deň zmizli a ľudia prišli o svoje investície.  
\subsection{Zhrnutie} 
Z toho čo som spomenul predtým je zjavné,  že pre naše zámery je okrajová burza ideálna. Keďže naše investície do burzy nie sú veľké, veľkými výhodami sú pre nás nízke poplatky a absencia veľkých hráčov, pričom vysoká likvidita je pre nás iba malou nevýhodou. S ostatnými rizikami a nevýhodami musíme rátať. Ktoré burzy(platformy) vyhovujú našim požiadavkám uvediem potom ako predstavím pojem algoritmického obchodovania a menu ktorá sa na týchto burzách vyskytuje.
\section{Algoritmické obchodovanie}
Algoritmus je špecifická množina jasne definovaných inštrukcií, ktorých cieľom je vykonať úlohu alebo proces. \\
Algoritmické obchodovanie (automatizované obchodovanie, obchodovanie black-box, alebo jednoducho algo obchodovanie) je proces používania počítačov, naprogramovaných nasledovať definovaný algoritmus, ktorý je usmernený na vykonanie obchodu s cieľom vytvárať zisk, a to s  rýchlosťou a frekvenciou, ktoré sú nemožné pre človeka - obchodníka. Deje sa tak v definovanej sade pravidiel, ktoré sú založené na načasovaní, cene a iných kritériách. Okrem ziskovej príležitosti pre obchodníka je algo-obchodovanie %na trh likvidnejšie a 
robí obchodovanie systematickejšie vylúčením emocionálneho vplyvu človeka na obchodné činnosti.\cite{Ba} \\
Aby mohlo prebiehať algoritmické obchodovanie musí mať burza API. API je prístup pre algoritmus na burzu.
\subsection{Backtesting} 
Backtesting je spôsob ako otestovať svoj algoritmus. Používajú sa historické dáta z búrz na otestovanie účinnosti algoritmu.
\section{Bitcoin} 
V poslednej dobe sa výrazne zvýšil počet búrz s takzvanými \uv{krypto menami} a jednou z najvýznamnejších je práve bitcoin.
\subsection{O bitcoine} 
Bitcoin umožňuje nový platobný systém s úplne digitálnymi peniazmi. Jedná sa o prvú decentralizovanú platbu siete \uv{peer-to-peer}, ktorá je poháňaná svojim užívateľom bez ústredného orgánu alebo sprostredkovateľov. Z užívateľského hľadiska je používanie bitcoinu skoro také jednoduché ako používanie internet bankingu. Bitcoin je prvá realizácia koncepcie nazvanej \uv{krypto meny}, ktorá bola prvýkrát popísaná v roku 1998 pánom Wei Dai a naznačovala myšlienku novej formy peňazí, ktorá používa šifrovanie na riadenie jej vzniku a transakcií, miesto centrálnych autorít.  Prvá špecifikácia bola publikovaná v roku 2009  Satoshi Nakamotom. 
\subsection{Spôsob získavania} 
Bitcoini môžete získať viacerými spôsobmi. Môže vám niekto nimi zaplatil, môžete ich kúpiť na burze alebo ich jednoducho dostanete. Ale existuje ešte jeden zaujímavý spôsob a to je ťažba. 
\subsubsection{Ťažba} 
Toto je spôsob, keď požičiate výpočtovou kapacitu svojho zariadenia na vedecké účely a za odmenu získate istý počet bitcoinov. 
 Sprostredkovatelia sú napríklad  Cgminer, Guiminer a BAMT. Mh/s je jednotka určujúca výpočtový výkon. Ukazuje koľko je vaše ťaženie schopné urobiť operácií za jednu sekundu. Platí, že čím viac, tým väčší zisk.
\subsection{Použitie} 
Bitcoin sa už v hojnej miere používa na internete. Podporuje ho už aj veľa známych obchodov. Dokonca už niektoré ne-internetové obchody zaviedli platenie bitcoinami. Na Slovensku máme už aj bitcoinový bankomat. 
 \\
Bohužiaľ, keďže bitcoin je slabo kontrolovateľný, často sa využíva na nelegálne obchody s drogami alebo na pranie peňazí. \cite{B}
\section{Vyhovujúce platformy} 
Ako som spomenul pri algoritmickom obchodovaní, potrebujeme aby burza mala svoje API. V tejto časti si predstavíme burzy, ktoré najviac vyhovujú naším požiadavkám. Sú to burzy s krypto-menami a hlavne s bitcoinom, lebo ten má pre nás veľmi dobré vlastnosti. 
\subsubsection{BITSTAMP} 
Túto burzu preferujeme najviac a na tejto burze chceme začať. Moja bakalárska práca sa pravdepodobne bude týkať hlavne tejto burzy. BITSTAMP ma veľmi prehľadné API, vyhovujúce transakčné poplatky a iné nám vyhovujúce vlastnosti. Transakčné poplatky sú od 0,5 percenta z transakcie, keď je transakcia do 500 dolárov až po 0,2 percenta keď je naša transakcia nad 120000 dolárov. Obchodovací pár je BTC/USD. \cite{Bit} 
\subsubsection{Bitfinex} 
Tu sa poplatky pohybujú od 0,1 až po 0,2 percenta z transakcie. Obchodovací pár je BTC/USD. Okrem bitcoinu sa dá obchodovať ešte z namecoinom a litecoinom. \cite{Bitf} 
\subsubsection{BTC-e} 
Poplatky sa transakciu sú od 0,2 do 0,5. Obchodovací pár je BTC/USD. \cite{BTC} 
\subsubsection{Kraken} 
Táto burza okrem bitcoinu obchoduje aj s inými krypto-menami, preto by sme túto burzu použili, keby by sme chceli prejsť na tieto meny. \cite{Kre} 
\section{Existujúce riešenie} 
\subsection{Tradewave} 
Toto existujúce riešenie vzniklo iba toto leto, tým pádom až potom ako sme začali rozvíjať našu myšlienku. Tradewave je veľmi podobný tomu, čo chceme vytvoriť my, dokonca v užívateľskom rozhraní je ešte lepší, má však niektoré, pre nás podstatné, nevýhody.  
\subsubsection{Čo je to Tradevawe?} 
Tradewave je veľmi pohodlný spôsob, ako si vytvoriť vlastní obchodovací algoritmus. Programátorský jazyk sa používa Python. Obchoduje práve na tých burzách, ktoré som spomínal v časti \uv{Vyhovujúce platformy}. Je kompletne automatizovaný. Obchody, grafy a záznamy zobrazuje v reálnom čase. Algoritmus môže bežať neustále na ich serveroch a tiež sa pohodlne spustiť a zastaviť. Dokonca má aj mailové upozorňovanie. Priamo na stránke Tradewave sa dá váš algoritmus aj jednoducho a rýchlo testovať na historických dátach.  \cite{Tw} 
\begin{figure}[!hbt]
\begin{center}
\includegraphics[width=1\textwidth]{trade}
\caption{Tradewawe}
\label{img:trade}
\end{center}
\end{figure}
\ref{img:trade}
\subsubsection{Vzhľadom k nášmu riešeniu} 
Toto riešenie je oproti nášmu veľmi užívateľský pohodlné. My sa nebudeme venovať užívateľskému rozhraniu, keďže s naším algoritmom a jeho testovaním budeme pracovať iba my. Ale nemôžme použiť nič z tohto existujúceho riešenia, pretože nemajú verejné kódy a hlavným dôvodom, prečo nemôžme Tradewave použiť je, že pri používaní tejto platformy sa všetko použité stáva aj ich vlastníctvom. Preto vzhľadom na fakt, že v budúcnosti by sme chceli na našom algoritme aj zarábať, si nemôžme dovoliť použiť Tradewave a tak odovzdať naše nápady a kódy niekomu inému. 
\section{Použité technológie} 
V tejto časti predstavíme technológie, ktoré budú použité pri programovaní algoritmu a testovania. 
\subsubsection{Ruby} 
Väčšina kódu bude naprogramovaná jazyku v Ruby. Ruby je dynamický, open source programovací jazyk so zameraním na jednoduchosť a produktivitu. Má elegantnú syntax, ktorú je prirodzene čítať a jednoducho písať. Ruby bol prvý krát uvedený v roku 1995 svojím tvorcom Yukihiro “Matz” Matsumoto. Matz skombinoval svoje obľúbené jazyky (Perl, Smalltalk, Eiffel, Ada, and Lisp). Ruby je vyvážením funkcionálneho a imperatívneho programovania. Svojej obľube sa teší až od roku 2006.\cite{Rb} 
\subsubsection{Python} 
Nejaké časti kódu sa môžu objaviť aj v jazyku Python. Python je interpretovaný, objektovo orientovaný, vysoko-úrovňový programovací jazyk s dynamickou sémantikou. Je postavený na dátových štruktúrach, v kombinácii s dynamickým písaním a dynamickými väzbami, aby bol atraktívny pre rýchly vývoj aplikácií. Rovnako môže byť použitý ako skriptovací jazyk alebo ako \uv{gloo language} na pripojenie existujúcich komponentov dohromady. Prvé uvedenie Pythonu je z decembra 1989 svojím tvorcom Guido van Rossum\cite{Pt} 
